\documentclass[12pt]{article}
\usepackage[english]{babel}
\usepackage[letterpaper,margin=1in]{geometry}
\usepackage[parfill]{parskip}
\usepackage{mathtools}
\usepackage{amssymb}
\usepackage{underscore}
\usepackage{hyperref}
\usepackage{graphicx}
\graphicspath{{img/}}
\frenchspacing
\author{Ariel Davis (azdavis), Jerry Yu (jerryy)}
\date{\today}
\title{15-418 Final Project Proposal: Portrait Mode}
\begin{document}
\maketitle

\section*{Summary}

We are going to build portrait mode with bokeh blur on NVIDIA GPUs.

Portrait mode has been popularized largely by high end DSLR cameras, where the
foreground of the subject is in focus and the background is blurred by bokeh
shapes. Recently, smartphone cameras without specialized hardware have gained
the ability to take portrait mode pictures by using software-only background
detection and manipulation methods, which is what we will implement.

\section*{Background}

There are two main stages to our application, extracting the background and
blurring.

We will extract the background using active contour model, known as the snake
method. Essentially, we will calculate an energy for each pixel in the image.
The energy is a weighted sum of the filtered intensity of the pixel and edge
value of a pixel, calculated with the Canny Edge detector or Marr Hildreth.
We will use the energy values to iteratively update snakes, a set of points
that move from the outer edge to the center based on the difference in energies.

Additionally, each snake has internal energy that prevents points from being
too far apart and prevents oscillations in the points.
At the termination of the algorithm, the snakes will have fit the foreground
of the image, which is where the energy difference is the highest.

The second step is to apply the bokeh blur. We will apply a filter in the shape
of a polygon across an image. Values are accumulated by shifting the image in a
shape and computing a weighted sum.

Repeat the following steps before snake converges, when each point finds a
energy difference above a threshold.
\begin{enumerate}
    \item
        Calculate energy values for each pixel
            \begin{enumerate}
                \item
                    Get grayscale image by averaging RGB values of each pixel.
                    (Highly parallelizable across pixels)
                \item
                    Calculate edges by convolving a gaussian filter across the
                    image. (Highly parallelizable across pixels)
            \end{enumerate}
    \item
        Move each point of snake based on energy value. (Parallelizable across
        point of snake)
    \item
        Compute internal energy of snake. (Requires synchronization within a
        snake)
\end{enumerate}

\begin{enumerate}
    \item
        Convolve the image with the bokeh filter. (Highly parallelizable across
        pixels)
\end{enumerate}

\section*{The Challenge}

TODO

\section*{Resources}

TODO

\section*{Goals and Deliverables}

TODO

\section*{Platform Choice}

TODO

\section*{Schedule}

TODO

\end{document}
