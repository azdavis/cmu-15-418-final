\documentclass[12pt]{article}
\usepackage[english]{babel}
\usepackage[letterpaper,margin=1in]{geometry}
\usepackage[parfill]{parskip}
\usepackage{mathtools}
\usepackage{amssymb}
\usepackage{underscore}
\frenchspacing
\author{Ariel Davis (azdavis), Jerry Yu (jyu)}
\date{\today}
\title{15-418 Final Project Proposal}
\begin{document}
\maketitle

\subsection*{Overview}
We have decided to build bokeh portrait mode in parallel.

Bokeh portrait mode has been popularized largely by DSLR cameras, where the
foreground of the subject is in focus and the background is blurred by bokeh
shapes. Up until recently, images with bokeh portrait mode had to be taken
from high end DSLR cameras that cost up to thousands of dollars. But now,
portrait mode has been integrated into smartphones, with the foreground
detection and blur done mainly with image processing. (Google Pixel 2) Many
smartphones in the market currently use multiple cameras (IPhone X) for
depth detection but our project will focus on the image processing method.

Portrait mode can be seperated into two main steps. The first is the segment
the given image into a foreground and background. Segmentation is done in order
to replicate the DSLR camera being able to capture foregrounds in focus and
backgrounds out of focus.
The second is the bokeh blur, which is applied to the background parts of the
image. A bokeh blur must be calculated with a polygon, in order to replicate the
fractions of light let through the specific apertures a DSLR camera.

\end{document}
