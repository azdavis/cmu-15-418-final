\documentclass[12pt]{article}
\usepackage[english]{babel}
\usepackage[letterpaper,margin=1in]{geometry}
\usepackage[parfill]{parskip}
\usepackage{graphicx}
\graphicspath{{img/}}
\usepackage{hyperref}
\frenchspacing
\author{Ariel Davis (azdavis), Jerry Yu (jerryy)}
\date{\today}
\title{15-418 Final Project Checkpoint 02}
\begin{document}
\maketitle

\section{Progress Review}

We have finished both the sequential C implementation and parallel CUDA
implementation of our algorithm.

We have verified that all three implementations produce the same, bit-for-bit
results on a given input image.

In order to avoid any encode/decode steps when working with JPEG, we modified
the algorithm to work with PPM, which is an image format we used in Assignment
1 and is much easier to work with.

We saw a significant amount of speedup between the sequential Python and
sequential C implementations. Then, after completing the CUDA implementation,
we saw more impressive speedups. In the CUDA implementation, we took advantage
of shared memory for our blur convolutions and synchronized threads between
the load and blur steps.

\section{Preliminary Results}

Below, we have summarized some results. Although the speedup from Python to C
is also impressive, we do not explictly include it in the table below because
the focus of the project is to consider the speedup gained specifically by
utilizing parallelism, not by choosing a ``fast'' language. We have also emited
some python results because they take too long to run.

\begin{tabular}{l|l|l|l|l}
    Image (size in px) & Python & C & Cuda & Speedup \\
    \hline
    Elephant 389x584 & 2m39s & 2.84s & 0.345s & 8.23x \\
    Tiger 1400x845 & - & 18.1s & 0.328s & 55.19x \\
    Large Elephant 2594x3888 & - & 134.37s & 1.040s & 129.17x \\
\end{tabular}

Time Breakdown in seconds for Large Elephant \\
\begin{tabular}{l|l|l|l}
    Task & C & Cuda & Speedup From Parallelism\\
    \hline
    Load Image & 0.5091 & 0.0132 & - \\
    Cuda Move Memory & - & 0.1734 & - \\
    Generate Mask & 0.1376 & 0.0963 & - \\
    Blur & 136.9151 & 0.2244 & 601.139x \\
    Write Image & 0.3073 & 0.4284 & -
\end{tabular}

\section{Goals}

We have essentially completed all the goals we set out to accomplish in our
proposal. However, we still have time left to work on the project.

We would like to try using OpenMP as a source of parallelism to achive speedup
in running our algorithm. We will revise the sequential C program to add OpenMP
pragmas, and compile sequential C and OpenMP-enabled C versions of the program
separately, as in Assignment 3.

We are interested as to what kinds of speedups we will see from sequential C to
OpenMP-enabled C. We expect there to be fairly significant speedups, but not on
the order of what we saw with CUDA, because our algorithm is basically various
image manipulations operations, and GPUs were originally designed to be able to
quickly execute such image manipulations.

\section{Revised Schedule}

\begin{tabular}{l|l}
    Date & Item \\
    \hline
    TODO & TODO \\
    TODO & TODO \\
    TODO & TODO \\
    TODO & TODO \\
    TODO & TODO
\end{tabular}

\section{Deliverables}

TODO

\section{Issues}

TODO

\section{Sample Results Gallery}

TODO

\end{document}
